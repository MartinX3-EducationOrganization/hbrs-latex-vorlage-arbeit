\chapter{Anhang}\label{ch:anhang}
\begin{figure}[H]
    \begin{ganttchart}[expand chart=\textwidth, hgrid, vgrid]{1}{14}
        \gantttitle{2024}{14} \\
        \gantttitle{Woche}{14} \\
        \gantttitlelist{1,...,14}{1} \\
        \ganttbar{\textit{Aufgabe 1}}{1}{2} \\
        \ganttlinkedbar{\textit{Aufgabe 2}}{3}{3} \\
        \ganttlinkedbar{\textit{Aufgabe 3}}{4}{7} \\
        \ganttlinkedbar{\textit{Aufgabe 4}}{7}{7} \\
        \ganttlinkedbar{\textit{Aufgabe 5}}{8}{8} \\
        \ganttlinkedbar{\textit{Aufgabe 6}}{8}{10} \\
        \ganttlinkedbar{\textit{Aufgabe 7}}{11}{11} \\
        \ganttlinkedbar{\textit{Aufgabe 8}}{11}{11} \\
        \ganttlinkedbar{\textit{Aufgabe 9}}{11}{11} \\
        \ganttlinkedbar{\textit{Aufgabe 10}}{11}{14}
        \label{fig:gantchart}
    \end{ganttchart}
    \caption{Gantt-Diagramm der detaillierten Arbeitsplanung}
    \begin{itemize}
        \item \textit{Aufgabe 1:} Einarbeitung in A
        \item \textit{Aufgabe 2:} Einarbeitung in B
        \item \textit{Aufgabe 3:} Entwicklung von As
        \item \textit{Aufgabe 4:} Bewerten von A
        \item \textit{Aufgabe 5:} Entwicklung von Interviews
        \item \textit{Aufgabe 6:} Interviewen der Probanden
        \item \textit{Aufgabe 7:} Zusammenführen der Ergebnisse
        \item \textit{Aufgabe 8:} Auswertung der Ergebnisse mittels B
        \item \textit{Aufgabe 9:} Finale Auswahl des passenden A
        \item \textit{Aufgabe 10:} Schreiben der Bachelorarbeit
    \end{itemize}
\end{figure}
