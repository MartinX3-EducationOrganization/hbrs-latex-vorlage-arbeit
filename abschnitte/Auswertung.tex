\chapter{Auswertung}\label{ch:auswertung}


\section{Flowchart-Diagramm}\label{sec:flowchart-diagramm}
\enquote{Prozesse sollten wann immer möglich grafisch modelliert werden!}
Um textuell beschriebene Prozesse mit zugehörigen Varianten und Negativfällen besser zu verstehen, werden diese visualisiert und durch Beschreibungen ergänzt.
Dazu eignen sich \enquote{Swimlane}-Diagramme.
Diese gehören zur Gruppe der Prozessablaufdiagramme.


\begin{figure}[H]
    Hier wird ein beispielhaftes Flussdiagramm aufgezeigt.
    \begin{center}
        \begin{tikzpicture}[node distance=2cm]
            \node (start) [startstop] {Start};

            \node (in1) [io, below of=start] {Input};
            \draw [arrow] (start) -- (in1);

            \node (pro1) [process, below of=in1] {Process 1};
            \draw [arrow] (in1) -- (pro1);

            \node (dec1) [decision, below of=pro1, yshift=-1.5cm] {Decision 1};
            \draw [arrow] (pro1) -- (dec1);

            \node (pro2a) [process, below of=dec1, yshift=-2cm] {Process 2a with a long text};
            \draw [arrow] (dec1) -- node[anchor=east] {yes} (pro2a);

            \node (out1) [io, below of=pro2a, yshift=-0.5cm] {Output};
            \draw [arrow] (pro2a) -- (out1);

            \node (subpro1) [subprocess, right of=dec1, xshift=2.5cm] {A very long subprocess};
            \draw [arrow] (dec1) -- node[anchor=south] {no} (subpro1);

            \node (stop) [startstop, below of=out1] {Stop};
            \draw [arrow] (subpro1) |- (stop);
            \draw [arrow] (out1) -- (stop);
        \end{tikzpicture}
        \caption{Beispiel Flowchart}\label{fig:beispiel-flowchart}
    \end{center}
\end{figure}

\begin{landscape}
    \begin{figure}[H]
        \begin{center}
            \begin{tabular}{|c|c||c|c|c|c|c|c|c|c|c||c||c|c|}
                \hline
                P & Q & $\neg$ & ( P & $\vee$ & Q ) & $\vee$       & ( $\neg$ & P & $\wedge$ & Q ) & $\equiv$ & $\neg$       & P \\ \hline \hline
                0 & 0 & 1      & 0   & 0      & 0   & $\mathbf{1}$ & 1        & 0 & 0        & 0   &          & $\mathbf{1}$ & 0 \\ \hline
                0 & 1 & 0      & 0   & 1      & 1   & $\mathbf{1}$ & 1        & 0 & 1        & 1   &          & $\mathbf{1}$ & 0 \\ \hline
                1 & 0 & 0      & 1   & 1      & 0   & $\mathbf{0}$ & 0        & 1 & 0        & 0   &          & $\mathbf{0}$ & 1 \\ \hline
                1 & 1 & 0      & 1   & 1      & 1   & $\mathbf{0}$ & 0        & 1 & 0        & 1   &          & $\mathbf{0}$ & 1 \\ \hline
            \end{tabular}
            \caption{Wahrheitstabelle: $\neg$ ( P $\vee$ Q ) $\vee$ ( $\neg$ P $\wedge$ Q ) $\equiv$ $\neg$ P}\label{fig:figureFormulaTable0}
        \end{center}
    \end{figure}
\end{landscape}